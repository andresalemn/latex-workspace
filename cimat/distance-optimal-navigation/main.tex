\documentclass{beamer}
\usepackage{siunitx,booktabs,xspace,multirow,graphicx,lipsum}
%\usepackage[spanish,mexico,es-noindentfirst]{babel}
\usepackage[table]{xcolor}
\usepackage[utf8]{inputenc}
\usepackage{tikz}
\usepackage{subcaption}
\usepackage[charter]{mathdesign}
\usetheme{Madrid}
\usepackage{tcolorbox}
\usepackage{hyperref}

\usefonttheme{serif}


\setbeamertemplate{frametitle}{
  \begin{beamercolorbox}[wd=\paperwidth,ht=1cm, sep=0.25cm]{frametitle}
    \begin{tikzpicture}[remember picture,overlay]
      \node[anchor=north east] at (current page.north east) {\includegraphics[height=0.55cm]{special_figures/logo_blanco.png}};
    \end{tikzpicture}
    \usebeamerfont{frametitle}\bfseries\insertframetitle
  \end{beamercolorbox}
}

\setbeamercolor{structure}{fg=beamer_color}
\setbeamertemplate{description item}[align left]
\setbeamercolor{description item}{fg=beamer_color}
\setbeamertemplate{caption}[numbered]
\setbeamercolor{frametitle}{bg=beamer_color,fg=white}
\setbeamercolor{author in head/foot}{bg=beamer_color,fg=white}
\setbeamercolor{title in head/foot}{bg=beamer_color,fg=white}
\setbeamercolor{date in head/foot}{bg=beamer_color,fg=white}
\setbeamercolor{titlelike}{parent=structure,bg=beamer_color,fg=white}
\setbeamertemplate{section in toc}[sections numbered]
\setbeamertemplate{subsection in toc}[subsections numbered]
\setbeamerfont{section in toc}{series=\bfseries}
\setbeamercolor{section in toc}{fg=black}
\setbeamercolor{caption name}{fg=blue}
\setbeamerfont{caption name}{series=\bfseries}
\setbeamerfont{caption}{size=\small}
\setbeamercolor{bibliography entry author}{fg=black}
\setbeamercolor{bibliography entry title}{fg=black} 
\setbeamertemplate{section in toc}{\inserttocsectionnumber.~\inserttocsection\par}
\setbeamertemplate{subsection in toc}{\hspace{1.5em}\inserttocsectionnumber.\inserttocsubsectionnumber~\inserttocsubsection\par}



\title[\textbf{Paper Presentation}]{\textbf{Distance-Optimal Navigation in an Unknown Environment Without Sensing Distances}}
\author[\textbf{Andrés Alemán}]{\textbf{Luis Andrés Alemán Sánchez}}
\date{\textbf{February 2026}}

\renewcommand{\maketitle}{
\vspace{0.5\baselineskip}

%\begin{center}
%    \textbf{MAESTRÍA EN CIENCIAS EN INGENERÍA MECATRÓNICA}
%\end{center}

\titlepage
}

% \AtBeginSection[ ]
% {
% \begin{frame}{Contenido de la presentación}
% \tableofcontents[currentsection,subsectionstyle = shaded]
% \end{frame}
% }

% \AtBeginSubsection[ ]
% {
% \begin{frame}{Contenido de la presentación}
% \tableofcontents[currentsection,currentsubsection]
% \end{frame}
% }

% Definir color
\definecolor{beamer_color}{RGB}{115,36,61} 
\definecolor{backframe_color}{RGB}{248, 248, 255}

\newcommand{\descmath}[1]{%
  \begingroup\color{beamer_color}\(#1\)\endgroup
}

\begin{document}

\begin{frame}
    \maketitle
    \centering
    \cite{tovar2007distanceoptimal}
\end{frame}

\begin{frame}{Contents}
    \tableofcontents
\end{frame}

\section{Sections of the Paper}

\subsection{Introduction \& Previous Work}

% --- Slide 1: Mini-Glossary (The Reference) ---
\begin{frame}{Mini-Glossary: Foundation Concepts}
    \begin{description}[labelwidth=-0.8cm]
        \item[Gap] A "Depth Discontinuity" – an obstacle or a corner that hides part of the environment.    
        \item[Metric Map] A "Blueprint" – knows exact lengths and angles.
        \item[Homotopy Class] An equivalence class of paths that can be continuously deformed into one another while keeping their endpoints fixed.
        \item[Topological Map] A "Subway Map" – knows how regions are connected, but not distances.
        \item[Simply Connected] A room with no pillars or holes (one continuous boundary).
        \item[ Information Space] Focuses on the information state ($\eta$), which is the history of all sensor readings and actions taken by the robot up to that point \cite{lavalle2006planning}.
        \item[Euclidean Optimality] The absolute shortest path between two points.
    \end{description}
\end{frame}

% --- Slide 2: The Hook ---
\begin{frame}{The Minimalist Challenge}
    \begin{tcolorbox}[colback=backframe_color,colframe=beamer_color,title=The Research Question]
        What is the minimal information required for a robot to navigate optimally?
    \end{tcolorbox}
    \vfill
    \begin{columns}[T]
        \begin{column}{0.48\textwidth}
            \textbf{Traditional Navigation}
            \begin{itemize}
                \item Requires exact $(x, y)$
                \item Measures distances/angles
                \item Needs high-power sensors (LiDAR/GPS)
                \item \textit{"More data is better"}
            \end{itemize}
        \end{column}
        \begin{column}{0.48\textwidth}
            \textbf{This Paper's Approach}
            \begin{itemize}
                \item No coordinates/compass
                \item No distance measurements
                \item Minimalist sensing
                \item \textit{"Information is the key"}
            \end{itemize}
        \end{column}
    \end{columns}
    \vspace{\baselineskip}
    Can a robot achieve \textbf{Euclidean Optimality} using only the sequence of \textbf{Gaps} it sees, without ever measuring a single distance?
\end{frame}

% --- Slide 3: Section I - The Information Paradox ---
\begin{frame}{Section I: Introduction}
    \vspace{-\baselineskip}
    \begin{tcolorbox}[colback=backframe_color,colframe=beamer_color,title=What are the assumptions?]
    Model the robot as a point moving in a \textbf{simply connected}* unknown planar environment. The robot has a gap sensor.
    \end{tcolorbox}
    \begin{columns}[c] % c = vertically centered
      \column{0.3\textwidth}
        \includegraphics[width=\linewidth]{figures/figure1_environments.png}
      \column{0.4\textwidth}
        \scriptsize{Fig. 1. (a)–(e) Environments which are indistinguishable to the robot (black disc) with limited sensing; however, it can still navigate optimally using the GNT shown in (f).}
    \end{columns}
    To characterize its environment, the robot builds a dynamic data structure, called the \textbf{Gap Navigation Tree (GNT)}, and is updated to mantain shortest-path information from the current position of the robot. 
    
    \vspace{\baselineskip}
    \scriptsize{*Multiply connected environments are adressed in a following section.}
\end{frame}

% --- Slide 4: Section II - Previous Work ---
\begin{frame}{Section II: Previous Work}
    The GNT is a \textbf{topological map} approach that bridges the gap between traditional approaches (achieving Euclidean optimality with minimalist sensors):
    \vspace{0.3cm}
    \begin{itemize}
        \item \textbf{Grid-Based Maps:} Discretize space into regular grids; high computational and memory burden.
        \item \textbf{Exact Geometric Representations:} Use precise coordinates/polygons; require \textbf{exact localisation} and perfect sensing.
        \item \textbf{Bug Algorithms:} Uses local information for constant-memory navigation but results in inefficient \textbf{non-optimal} paths.
    \end{itemize}
    \vspace{0.3cm}
    \begin{table}[]
        \footnotesize
        \centering
        \begin{tabular}{|l|l|l|}
        \hline
        \textbf{Representation} & \textbf{Sensing Level} & \textbf{Distance Optimality} \\ \hline
        Grid / Geometric & Dense / Metric & Excellent (Global) \\ \hline
        Bug Algorithms & Minimalist & Poor \\ \hline
        \rowcolor[gray]{0.9} \textbf{GNT (Proposed)} & \textbf{Minimalist} & \textbf{Optimal*} \\ \hline
        \end{tabular}
    \end{table}
    \scriptsize{*With limited sensing, global optimality is impossible; GNT achieves local optimality within a \textbf{homotopy class}.}
\end{frame}

% --- Slide 5 --- Section II:  Visibility and Minimalism ---
\begin{frame}{Visibility \& Minimalism}
    \begin{itemize}
        \item \textbf{Minimalist Sensing Philosophy:}
        One can imagine an abstract sensor providing minimal ideal information, physically implemented by a subideal sensor.
        \item \textbf{The Visibility Region ($V(x)$):}
        The core foundation of this work. It stems from famous problems like:
        \begin{itemize}
            \item \textbf{Art-Gallery Problem:} How many cameras/guards are needed to see an entire room?
            \item \textbf{Aspect Graphs:} Groups together all viewpoints that see the same "combinatorial structure" of an object.
        \end{itemize}
        \item \textbf{Visual Events:} 
        A term used to describe the \textit{exact moment} the Visibility Region changes (crossing a "Visibility Cell" boundary). Our GNT is built entirely on these events.
        \item \textbf{Prior Data Structures:}
        A similar structure to the GNT exists in previous work \cite{aronov1998visibility}, but with a \textbf{critical difference}:
        \begin{itemize}
            \item \textbf{Previous Work:} Assumed \textit{perfect knowledge} of the map and \textit{exact localization}.
            \item \textbf{This Paper:} Assumes \textbf{unknown environments} and \textbf{no localization}.
        \end{itemize}
    \end{itemize}
\end{frame}

\subsection{Robot Model}

% --- Mini-Glossary for Section III ---
\begin{frame}{Mini-Glossary: Section III Notation}
    \begin{description}[labelwidth=-0.8cm]
        \item[\descmath{R,\, \partial R}] The environment and its boundary (the walls).
        \item[\descmath{G(x)}] The \textbf{Gap Sequence} observed at position $x$.
        \item[\descmath{[g_1, \dots, g_k]}] Brackets and ellipsis denote a \textbf{Cyclic Order}
        \item[\descmath{\mathrm{chase}(g)}] The \textbf{Motion Primitive} (Align $\rightarrow$ Move $\rightarrow$ Feedback).        
        \item[\descmath{\tau : [0, 1] \to R}] \textbf{Path}: The robot's movement over \textit{space} (normalized geometry). 
        \item[\descmath{\tau : [0, t_f] \to R}] \textbf{Trajectory}: The robot's movement over \textit{time} ($t_f$ is final time).
    \end{description}
    \vfill
    \footnotesize \textit{Note: Since the robot moves at \textbf{unit speed}, time and distance are equivalent, allowing $\tau$ to represent both trajectory and path.}
\end{frame}

% --- Slide 5: Transition to the Model ---
\begin{frame}{Section III: The Robot Model}
    \textbf{The Point Robot:}
    \begin{itemize}
        \item No Coordinates $(x, y)$
        \item No Compass / Orientation
        \item No Odometer (doesn't know how far it moves)
    \end{itemize}
    \vspace{0.3cm}
    \textbf{The Environment ($R$):}
    \begin{itemize}
        \item Unknown planar world.
        \item Boundary ($\partial R$) is "piecewise-analytic" (smooth curves + corners).
    \end{itemize}
    \vspace{0.3cm}
    \textbf{The Strategy:}
    Build a data structure purely from \textbf{Visual Events} (Gaps appearing, merging, or splitting).
\end{frame}

% --- Slide: The Gap Sensor (Logic) ---
\begin{frame}{III.A. The Gap Sensor}
    \begin{itemize}
        \item \textbf{The Core Function:} Detects "breaks" in the boundary $\partial R$ where depth is discontinuous.
        \item \textbf{Identity Tracking:} The robot maintains a unique label for each gap $g_i$ as long as it remains visible.
        \item \textbf{Output $G(x) = [g_1, \dots, g_k]$:}
        \begin{itemize}
            \item Reports the \textbf{order} of gaps, not their distance or angle.
            \item Since it is cyclic, $[g_1, g_2, g_3] = [g_2, g_3, g_1]$.
        \end{itemize}
    \end{itemize}
    \vspace{\baselineskip}
    \begin{block}{The "Correspondence" Assumption}
        The robot is fast enough to "watch" gaps move; it never loses track of which label belongs to which physical corner.
    \end{block}
\end{frame}

% --- Slide: The Gap Sensor (Visuals) ---
\begin{frame}{Visualizing the Sensor}
    \vspace{-\baselineskip}
    \begin{columns}[T]
        \begin{column}{0.5\textwidth}
            \begin{center}
            \includegraphics[width=\linewidth]{figures/figure2_visibility.png}
               \scriptsize{Fig. 2. Robot’s view of the environment}
            \end{center}
            \small
            \begin{itemize}
                \item \textbf{On the left:} The Visibility Region $V(x)$ in $R$.
                \item \textbf{On the right:} The abstract representation $G(x)$.
                \item The 2D world is reduced to a 1D sequence.
            \end{itemize}
        \end{column}
        \begin{column}{0.5\textwidth}
            \begin{center}
        \includegraphics[width=\linewidth]{figures/figure3_tangent.png}
               \scriptsize{Fig. 3. Gaps from the environment’s boundary.}
            \end{center}
            \small
            \begin{itemize}
                \item Works for smooth curves (\textbf{3a}) and sharp corners (\textbf{3b}).
                \item A gap exists wherever the robot's line of sight "grazes" $\partial R$.
            \end{itemize}
        \end{column}
    \end{columns}
\end{frame}

% --- Slide: Motion Primitives ---
\begin{frame}{III.B. Motion Primitives}
    \begin{itemize}
        \item {The robot motions are expressed as a sequence of \textbf{motion primitives}, which are described solely in terms of information from the sensor.}
        \item {A gap chasing motion primitive is denoted as chase(g)}
    \end{itemize}
    \vspace{\baselineskip}
    \begin{columns}[T]
        \begin{column}{0.5\textwidth}
            \footnotesize {Execution of chase(g):}
            \begin{enumerate}
                \item Align heading with gap $g$.
                \item Move forward at \textbf{unit speed}.
                \item Use \textbf{feedback} to adjust heading as $g$ shifts.
            \end{enumerate}
            \vspace{0.3cm}
			\footnotesize{The robot navigation capabilities are simple:}
            \begin{itemize}
                \item {Movement towards boundary components $\&$ wall-following.}
                \item {The chase$(g)$ terminates when $g$ disappears from $G(\tau(t))$.}
            \end{itemize}
        \end{column}
        \begin{column}{0.5\textwidth}
            \begin{center}
                \includegraphics[width=\linewidth]{figures/figure4_chase.png}
                \scriptsize{Fig. 4. The chase($\cdot$) motion primitive.}
            \end{center}
            \footnotesize
            The path $\tau$ isn't always a straight line; the robot "slides" along boundaries to stay locked on $g$.
        \end{column}
    \end{columns}
\end{frame}

\subsection{Gap Navigation Tree}

% --- Mini-Glossary for Section IV ---
\begin{frame}{Mini-Glossary: GNT Formalism}
	\begin{description}[labelwidth=-0.8cm]
		\item[\descmath{s \in [0, 1]}] Normalizes the path $\tau(s)$ from start to end.
		\item [\descmath{G(\tau(s))}] Denotes the cyclic sequence of all currently visible gaps detected by the gap sensor when the robot is at configuration $\tau(s)$
	\end{description}
\end{frame}

\begin{frame}{Section IV: Gap Navigation Tree}
	\begin{itemize}
		\item Suppose that the robot moves along any path $\tau : [0,1] \to R$. Consider the information obtained from the gap sensor.
		\item For every $s \in [0,1]$, a cyclic sequence $G(\tau(s))$ of gaps is observed.
		\item Here, we define a compact representation of information that is relevant for optimal navigation and appears in $G(\tau(s))$ for all $s \in [0,1]$.
	\end{itemize}
	\begin{center}
		\includegraphics[width=\linewidth]{figures/figure4_5_gnt.png}
		\cite{nasirGNT}
	\end{center}
\end{frame}

% --- Slide: IV.A Compressing Sensor History ---
\begin{frame}{IV.A: Compressing Sensor History}
    \begin{itemize}
	\item \textbf{Rooted Tree:} Compresses the history of $G(\tau(s))$. The root is the robot’s current position.
	\begin{itemize}
		\item Each nonroot vertex is a gap appearing in $G(\tau(s))$ for some $s \in [0,1]$.
		\item Children of the root are exactly the gaps in $G(\tau(1))$, cyclically ordered as in $G(\tau(1))$.
		\item All other vertices are gaps that appeared for some $s<1$ but not in $G(\tau(1))$ due to merging.
	\end{itemize}
        \item \textbf{Vertex Classification:}
        \begin{itemize}
            \item \textbf{Primitive:} A leaf that disappears when chased (no hidden regions).
            \item \textbf{Non-Primitive:} A leaf that splits when chased (hides other gaps).
        \end{itemize}
        \begin{block}{}
            A GNT is \textbf{Complete} if and only if all of its leaf vertices are \textbf{primitive}.
        \end{block}
        \item \textit{Insight:} A complete tree means the robot has explored every hidden "nook" in the environment.
    \end{itemize}
\end{frame}

% --- Slide: The Three Notations of Chase ---
\begin{frame}{Defining the "Chase" Actions}
    The paper uses three specific notations to describe movement:
    \vspace{0.4cm}
    \begin{itemize}
	        \item $\text{chase}(g)$: A \textit{single} motion primitive targeting a visible gap $g$.
	        \item \textbf{$\mathbf{chase}(g)$ (Bold):} A \textit{sequence} of primitives. Used to reach a gap $g$ hidden deep in the tree by chasing its ancestors first.
	        \item $\text{chase}(\cdot)$: Refers to the \textit{abstract concept} or class of motion primitives regardless of the specific gap.
	    \end{itemize}
\end{frame}

% --- Slide: IV.B & C Critical Events & Geometry ---
\begin{frame}{IV.B. Critical Events and GNT Construction}
	\begin{columns}[T]
		\begin{column}{0.4\textwidth}	
			\begin{itemize}
				\small
				\item Each time $t$ at which a change in $G(\tau(t))$ occurs corresponds to a \textbf{critical event}. 
				\item This requires updating the GNT.
				\item There are four different kinds of critical events.
				\end{itemize}
	    \begin{enumerate}
	    	\small
	        \item \textbf{Appearance:} New gap $g$ added to root.
	        \item \textbf{Merge:} $g_1, g_2$ become children of a new vertex.
	        \item \textbf{Disappearance:} Primitive leaf is removed.
	        \item \textbf{Split:} Non-primitive vertex divides.
	    \end{enumerate}
		\end{column}
		\begin{column}{0.6\textwidth}	
			\center
	    	\includegraphics[width=0.8\linewidth]{figures/figure5_updates.png}\\
		    \scriptsize{Fig. 5. Updates in the GNT. The relevant inflection or bitangent complement is indicated with a gray line segment.}
	    \end{column}
	\end{columns}
\end{frame}

\begin{frame}{Mini-Glossary: Geometry behind GNT}
	\begin{block}{}
		Critical events are determined by \textbf{generalized inflections} and \textbf{generalized bitangents} of the environment boundary ($\partial R$).
	\end{block}
% 	\vspace{\baselineskip}
	\begin{description}[labelwidth=-0.8cm]
		\item[\descmath{L}] (The Partitioning Line) A straight line that divides the local space into two half-planes. In the GNT model, $L$ represents the "critical threshold" or "sight-line."
		\item[\descmath{I}] (Primary Boundary Segment) A connected, open piece of the wall ($\partial R$). It is used to define both inflections and bitangents.
		\item[\descmath{J}] (Secondary Boundary Segment): A separate, disjoint piece of the wall ($\partial R$). It is used specifically to define bitangents Connected segments of the boundary.
		\item[{Inflection}] Trigger for Appearance and Disappearance of Gaps.
		\item[{Bitangent}] Trigger for Merging and Splitting of Gaps.
	\end{description}
\end{frame}

\begin{frame}{IV.C. Geometric Interpretation of the GNT}
	\vspace{-\baselineskip}
    \begin{columns}[T]
		\begin{column}{0.45\textwidth}	
			\vspace{0.2cm}
			\begin{itemize}
				\item \textbf{Inflections:} A geometric threshold where a boundary segment ($I$) crosses its own tangent line ($L$) marking the exact moment a gap appears or disappears from the robot's view.
				\item \textbf{Bitangents:} A "bridge" where a single line ($L$) is tangent to two separate boundary segments ($I$) and ($J$), causing two gaps to merge into one or one gap to split into two.
			\end{itemize}
		\end{column}
		\begin{column}{0.55\textwidth}
			\begin{center}
		    	\includegraphics[width=\linewidth]{figures/figure6_critical_events.png}\\
		    	\scriptsize{Fig. 6. Critical events. (a) Appearance and disappearance of gaps occur when the robot crosses inflection rays. (b) Splits and merge occur by crossing bitangent complements.}\\
	    		\vspace{\baselineskip}
	    		\normalsize
		    	\textbf{Lemma 1:} Let $g1$ and $g2$ be two gaps that merge into gap $g3$. When $g3$ splits, $g1$ and $g2$ appear at the same angular position in $R$ at the time of the merge, independently from the robot’s motion.		    	
			\end{center}
		\end{column}
	\end{columns}
\end{frame}

% --- Slide: IV.D The Gap-Based Roadmap ---
\begin{frame}{Section IV.D: The Gap-Based Roadmap ($S$)}
	\vspace{-\baselineskip}
	\begin{columns}[T]
		\begin{column}{0.6\textwidth}
			\begin{center}
				\includegraphics[width=\linewidth]{figures/figure7_chase_termination.png}\\
				\scriptsize{Fig. 7. Every chase($\cdot$) motion primitive terminates, either with a disappearance or with a split critical event. The black disc represents the position of the robot when the chase($\cdot$) motion primitive is issued. After reaching $\partial R$, the robot moves tangentially to $\partial R$, until (a) an inflection ray or (b) a bitangent complement is crossed.}\\
			\end{center}
		\end{column}		
		\begin{column}{0.4\textwidth}	
			\vspace{0.2cm}
		    \begin{itemize}
		        \item \textbf{Definition:} $S$ is the set of all points reached by executing a sequence of $\text{chase}(\cdot)$ primitives following critical events.
		        \item \textbf{Termination:} Every $\text{chase}(g)$ is guaranteed to terminate because the robot must eventually cross an inflection or bitangent.
		    \end{itemize}
    	\end{column}
	\end{columns}
	\vspace{\baselineskip}
	\normalsize
	\textbf{Lemma 2:}  Termination of chase $(g)$ is guaranteed for any $g \in G(x)$ and any $x \in R$ and is caused by only two possible critical	events: disappearance or splitting of $g$.
\end{frame}

% --- Slide: IV.E Constructing a Complete GNT ---
\begin{frame}{Section IV.E: Constructing a Complete GNT}
    \begin{columns}[T]
        \begin{column}{0.6\textwidth}
            \textbf{The Algorithm:}
            \begin{enumerate}
                \item Identify a non-primitive leaf vertex $g$.
                \item Execute $\mathbf{chase}(g)$.
                \item Update tree based on splits or disappearances.
                \item Repeat until all leaves are primitive.
            \end{enumerate}
            \vspace{0.3cm}
            \textbf{Lemma 3:} The procedure of iteratively chasing nonprimitive leaves terminates with a resulting complete GNT.\\
           	\vspace{\baselineskip}
 			\scriptsize{Fig. 8. Building the GNT. (a) The thin lines show the places where gap critical events are triggered. The robot chases nonprimitive gaps from (a) to (f), updating the GNT accordingly (refer to the text), until all of the leaf vertices are primitive. Squares and circles denote primitive and nonprimitive vertices, respectively.}
        \end{column}
        \begin{column}{0.35\textwidth}
       		\vspace{-\baselineskip}
            \begin{center}
				\includegraphics[width=0.9\linewidth]{figures/figure8_GNT.png}\\
            \end{center}
        \end{column}
    \end{columns}
\end{frame}

\subsection{Optimal Navigation}

% --- Mini-Glossary for Section V ---
\begin{frame}{Mini-Glossary: V. Optimal Navigation}
	\textbf{Crucial Assumption:} The exploration phase is complete. The robot now possesses a 'Complete GNT'*.
	\vspace{\baselineskip}
	\begin{description}[labelwidth=-0.8cm]
		\item[$p, q$] Points in the environment $R$ (Start and Goal).
		\item[$n$] The number of \textbf{inflections} (geometric complexity of the room).
		\item[$u_i$] A specific segment of the boundary $\partial R$ associated with a gap.
		\item[\descmath{U = (u_1, \dots, u_n)}] Sequence of maximal connected intervals of the boundary $\partial R$ that the robot traverses (in order) in the shortest path from $p$ to $q$.
      \item[$L, O$] \textbf{Landmarks} (fixed) and \textbf{Objects} (movable) in the tree.
	\end{description}	
	\vspace{\baselineskip}
	\footnotesize *Note: In simple connected rooms, completeness is guaranteed. The specific 'issues' that arise in rooms with pillars—and how to solve them—will be revealed in Section VI.
\end{frame}

% --- Slide: V.A Moving Along the Roadmap ---
\begin{frame}{Section V.A: Moving Along the Roadmap}
	\begin{itemize}
		\item \textbf{Lemma 4:} Let $H=(g_1, g_2, ..., g_n)$ be a sequence of gaps, in
			which $g_i$ is the gap chased when the robot traverses the interval $u_i \in U$. The path generated by chasing iteratively the sequence $H$ is the shortest path between $p$ and $q$.
		\item \textbf{Theorem 1:} If $R$ is simply connected and the robot is at a point in $S$, then the path encoded in the GNT between the root and any point $q \in S$ is \textbf{globally optimal} in Euclidean distance.
		\item \textbf{The "Taut String" Logic:} 
		\begin{itemize}
			\item The shortest path is a sequence of straight lines.
			\item These lines are tangent to $\partial R$ at specific intervals $u_i$.
			\item Chasing the gap sequence $H = (g_1, \dots, g_n)$ physically forces the robot to follow these tangent points.
		\end{itemize}
	\end{itemize}
	\begin{center}
		\textit{The robot follows the "taut string" without ever measuring its length.}
	\end{center}
\end{frame}


%		\item[$L, O$] \textbf{Landmarks} (fixed) and \textbf{Objects} (movable) in the tree.

% --- Slide: V.B Complexity ---
\begin{frame}{Section V.B: Complexity}
	\begin{description}
		\item[$O(n)$] \textbf{(Big-O Notation):} Represents linear asymptotic complexity.
	\end{description}
	 It means the resources (memory or time) needed by the robot grow in direct proportion to the complexity of the room.
	\vspace{0.4cm}
	\begin{itemize}
		\item \textbf{Space Complexity $O(n)$:} The GNT requires memory proportional to the number of inflections, not the size of the room.
		\item \textbf{Construction Complexity:} The construction of the GNT cannot take more than $O(n)$ gap-chasing motion commands.
		\item \textbf{Query Complexity:} A query for \textbf{chase$(\cdot)$} takes in the worst case $O(n)$ time.
	\end{itemize}
	\vfill
	\begin{block}{Efficiency Note}
		This allows the robot to operate in large environments with very limited computational power.
	\end{block}
\end{frame}

% --- Slide: V.C Traveling Anywhere ---
\begin{frame}{Section V.C Traveling Anywhere in $R$}
	\begin{columns}[T]
		\begin{column}{0.6\textwidth}
			\textbf{Moving to Targets:}
			\begin{itemize}
				\item \textbf{$\text{chase}(l)$:} Optimal sequence of moves to reach a landmark.
				\item \textbf{$\text{chase}(o)$:} Optimal moves to reach a specific object.
			\end{itemize}
			\vspace{0.3cm}
			\textbf{Theorem 2:} The extended GNT encodes a path to any object or landmark in the environment from the current position of the robot.\\
			\textbf{Corollary 1:} 
			The motions $\textbf{chase}(l)$ and $\textbf	{chase}(o)$ lead to distance-optimal paths between \textit{any} possible pair of positions in $R$.
		\end{column}
		\begin{column}{0.4\textwidth}
			\footnotesize
			A fifth critical event is included, which corresponds to the appearance of an object or landmark in $G(x)$.
			\begin{center}
            \includegraphics[width=\linewidth]{figures/figure9_objects.png}
			\scriptsize{Fig. 9. Encoding objects in the GNT. When the triangular object hides behind the gap, we associate such an object with the gap. The gap encodes the last time the object was visible (the object is hidden behind the gap).}
			\end{center}
		\end{column}
	\end{columns}
	\vspace{\baselineskip}
	\scriptsize
	When an object is hidden behind a gap, it is stored as a child of that gap in the tree. To find it, the robot simply "unfolds" the tree.
\end{frame}

\subsection{Multiply Connected Environments}
\subsection{Implementation}
\subsection{Conclusions \& Discussion}

\section{References}

\begin{frame}[allowframebreaks]{References}
\bibliographystyle{ieeetr}
\bibliography{biblio}
\end{frame}

\begin{frame}
    \maketitle
\end{frame}

\end{document}