\documentclass[a4paper,12pt,oneside,pdflatex,italian,final,twocolumn]{article}

\usepackage[utf8]{inputenc}
\usepackage{parallel}
\usepackage{siunitx}
\usepackage{booktabs}
\usepackage{fancyhdr}
\usepackage[export]{adjustbox}
\usepackage[top=1.0in,bottom=0.8in,left=0.7in,right=0.7in]{geometry}
\usepackage{libertine}
\renewcommand*\familydefault{\sfdefault} 
\usepackage[T1]{fontenc}
\usepackage{amsmath}

\title{HERMES MODULE}
\author{Inbiodroid}
\date{August 2025}

\begin{document}

% --- Cover Page ---
\begin{titlepage}
    \centering
    \vspace*{2.5cm}

    \includegraphics[width=0.4\textwidth]{figures/logo.png}\par
    \vspace{1cm}

    {\Huge \textbf{HERMES MODULE}}\par
    \vspace{0.5cm}
    {\LARGE Inbiodroid}\par
    \vspace{0.5cm}

    % Insert additional image here
    \includegraphics[width=0.4\textwidth]{figures/hermesCAD.png}\par
    \vspace{0.5cm}

    {\Large \textbf{Versión: v1.0}}\par
    \vspace{0.5cm}
    {\large Agosto de 2025}\par
    \vfill

    {\large Hoja Técnica}\par
\end{titlepage}

% --- Start of main content ---
\pagestyle{fancy}
\lhead{Inbiodroid}
\rhead{HERMES MODULE v1.0}
\fancyfoot[C]{\thepage}

\onecolumn

% --- Header Figure ---
\begin{figure}[h!]
\begin{minipage}{0.4\textwidth}
\centering
\includegraphics[width=.8\textwidth,center]{figures/logo.png}
\end{minipage}
\hfill
\begin{minipage}{0.4\textwidth}
\centering
\Huge \textbf{HERMES v1.0}
\end{minipage}
\end{figure}

% --- Resumen Section ---
\begin{figure}[h!]
\begin{minipage}{0.47\textwidth}
\section{Resumen}
\begin{itemize}
    \item Módulo para Captura de Movimiento
    \item Retroalimentación Háptica (Vibrador)
    \item Procesamiento de Sensores (9-DOF IMU)
    \item Comunicación Inalámbrica (ESP-32)
    \item +4 horas en Operación
\end{itemize}
\end{minipage}
\hfill
\begin{minipage}{0.47\textwidth}
\centering
\includegraphics[width=0.7\textwidth,right]{figures/hermesCAD.png}
\end{minipage}
\end{figure}

% --- Rest of the document ---
\vspace{-2em}
\section{Descripción General}
\begin{itemize}
\item El módulo Hermes forma parte del sistema de telepresencia Prometheus 3.0, capaz de teleoperar un robot humanoide de 20 DOF. 
\item Mediante una cadena de estos módulos colocados en el tren superior de una persona, Hermes está diseñado para capturar sus movimientos en tiempo real.
\item Arquitectura de Dispositivos Maestro - Esclavos: Se compone de dos tipos de unidades:
    \begin{itemize}
    \item Unidad `Stella` (Orquestador Central): Es la unidad principal del sistema, que se lleva en el pecho del operador. Recibe y centraliza los datos de todas las unidades Satelle.
    \item Unidades `Satelle` (Recolectores de Datos): Se colocan en las extremidades del operador (hombros, codos y muñecas).
    \end{itemize}
\end{itemize}

\vspace{-1.5em}
\begin{center}
\includegraphics[width=0.7\textwidth]{figures/hermesChain.png}
\end{center}
\vspace{-2em}

\section{Funcionamiento}
\begin{itemize}
 \item Comunicación:
 \begin{itemize}
    \item Cada Satelle lee su sensor IMU para determinar su orientación 3D (Cuaternión).
    \item Los datos de cada Satelle se envían de forma inalámbrica a la unidad Stella (paquetes UDP).
    \item La unidad Stella calcula la pose cartesiana (posición y rotación) deseada para el efector final de los brazos y además calcula las rotaciones del torso del robot humanoide.
 \end{itemize}
\end{itemize}

\vspace{-1em}

%\begin{center}
\begin{minipage}{0.65\textwidth}
\begin{itemize}
    \item Captura del movimiento para los brazos:
    \begin{itemize}
    \item Para cada brazo se utilizan tres módulos Satelle.
    \item Los primeros dos (hombro y codo) se emplean para determinar la posición cartesiana deseada en el efector final $P_\text{final}$.
    \item El tercer módulo (muñeca) se utiliza para calcular la orientación del efector final $R_{final}$.
    \item La pose cartesiana se compone de $ \begin{bmatrix} P_{final} & R_{final}\end{bmatrix}$
    \item[] $Pose_\text{} =\begin{bmatrix} x_{P} & y_{P} & z_{P} & w_{R} & x_{R} & y_{R} & z_{R}\end{bmatrix}$.
    \end{itemize}
 \end{itemize}
\end{minipage}%
\begin{minipage}{0.25\textwidth}
\includegraphics[width=\linewidth]{figures/cal.png}
\end{minipage}
%\end{center}
\begin{itemize}
    \item Captura del movimiento para el torso:
    \vspace{-0.5em}
    \begin{itemize}
    \item Stella el módulo encargado de generar las rotaciones para el torso del robot Prometheus por medio de ángulos de euler.
    \item Una vez calibrado, Stella genera los movimientos son de Yaw (rotación de izquierda a derecha) y Pitch (inclinación hacia adelante o hacia atrás).
    \end{itemize}
\end{itemize}

\section{Especificaciones Técnicas}
\centering
\begin{tabular}{ccc}
\toprule
 & Valor & Unidad \\
\midrule
Peso & 75 & $g$ \\
Dimensiones & 65*58*34 & $mm*mm*mm$ \\
n° de entradas & 2 & $--$ \\
Grados de Libertad & 9 & $DOF$ \\
Alimentación del Módulo & 5 & $V$ \\
Corriente mínima de Carga & 2 & $A$ \\
Temperatura de Operación & 15 - 40 & $C$ \\

\bottomrule
\end{tabular}

\raggedright

\section{Interfaz}
\begin{center}
\begin{minipage}{0.5\textwidth}
\centering
\begin{tabular}{lcr}
\toprule
Núm & Interfaz \\
\midrule
1 & Puerto C (Power 5V)  \\
2 & Puerto Jack (Periféricos)  \\
3 & Botón (Calibración y Batería)   \\
4 & Switch de Encendido y Apagado  \\
5 & LED de Estados (Multiusos) \\
\bottomrule
\end{tabular}
\end{minipage}%
\begin{minipage}{0.45\textwidth}
\centering
\includegraphics[width=\linewidth]{figures/hermesNumbers.png}
\end{minipage}
\end{center}

\section{Medidas}
\centering
\begin{figure}[h!]
\includegraphics[width=\textwidth]{figures/measures.png}
\end{figure}

\end{document}
